\documentclass[a4paper,10pt]{article}

\usepackage{amsmath}

\begin{document}

\title{Optimization of Projectile Range}
\author{Giulio Procopio}
\date{January 2024}

\maketitle

\section{Introduction}

Given a projectile
thrown with initial velocity $\vec{v_0}$
from a height $y_0$,
what is the optimal angle $\theta^*$
between such velocity and the horizontal plane
in order to maximize its range?

We will show that the optimal angle is, more generally,
not always $45^\circ$;
such angle is indeed dependent
from the initial state of the projectile
(i.e. its initial velocity and height).

\section{Initial State}

Let's consider a projectile
with initial velocity $\vec{v_0}$.
We'll study its motion in the plane $x,y$
from its initial position $(0,y_0)$ at time $t_0$, with $y_0>0$,
to the ground.

As such, $\vec{v_0} = (v_{0x},v_{0y})$ with $v_{0x}>0$ and $v_{0y}>0$.
The angle $\theta$ between $\vec{v_0}$ and the horizontal plane is
$\theta = \arctan\left(\frac{v_{0y}}{v_{0x}}\right)$.

The projectile will be subject to an uniform gravitational acceleration
$\vec{g}$ with $g\approx9.81\frac{m}{s^2}$ pointing downwards.
We'll neglect other forces such as air resistance.

\section{Equations of Motion}

The equations of motion for the projectile are

\begin{equation}
    \begin{cases}
        x(t) = v_{0x}t\\
        y(t) = y_0 + v_{0y}t - \frac{1}{2}gt^2
    \end{cases}
    , \quad t\geq t_0 = 0
\end{equation}

From the system above we can derive the 
function of the projectile's trajectory $y(x)$.
Remembering that $v_{0x} = v_0\cos\theta$ and $v_{0y} = v_0\sin\theta$,
we get

\begin{equation}
    y(x) = y_0 + \tan\theta\cdot x - \frac{g}{2v_0^2\cos^2\theta}x^2
\end{equation}

Let $r$ be the range of the projectile.
The point of contact with the ground is given by $(r,0)$, thus

\begin{equation}
    \begin{aligned}
        y(r) = 0
        \implies r & = \frac{
            -\tan\theta
            + \sqrt{
                \tan^2\theta
                + \frac{
                    2gy_0
                }{
                    v_0^2\cos^2\theta
                }
            }
        }{
            -\frac{
                g
            }{
                v_0^2\cos^2\theta
            }
        }\\
        & = \frac{
            v_0^2
        }{
            g
        }
        \left(
            \sin\theta
            + \sqrt{
                \sin^2\theta
                + \frac{
                    2gy_0
                }{
                    v_0^2
                }
            }
        \right)
        \cos\theta
    \end{aligned}
\end{equation}

For the sake of simplicity, 
let $A = \frac{v_0^2}{g}$ 
and $B = \frac{2gy_0}{v_0^2}$.
So that, writing $r$ as a function of $\theta$,

\begin{equation}
    r(\theta) = A\left(
        \sin\theta
        + \sqrt{
            \sin^2\theta
            + B
        }
    \right)
    \cos\theta
\end{equation}

\section{Degenerate Case $y_0=0$}

Let's first consider the degenerate case $y_0=0$.
The term $B$ goes to zero, thus $r(\theta)$ simplifies to

\begin{equation}
    r(\theta) = A\sin2\theta
\end{equation}

The range is maximized when the sine is maximum,
i.e. $\max r(\theta) = r(\theta^*) = A$ when $\theta^* = \frac{\pi}{4}$.

\section{General Case $y_0>0$}

Considering the general case $y_0>0$,
we evaluate the derivative of $r(\theta)$

\begin{equation}
    \begin{aligned}
        \frac{dr}{d\theta} & = A\left[
            -\sin\theta\left(
                \sin\theta
                + \sqrt{
                    \sin^2\theta
                    + B
                }
            \right)
            + \cos\theta\left(
                \cos\theta
                + \frac{
                    \sin\theta\cos\theta
                }{
                    \sqrt{
                        \sin^2\theta
                        + B
                    }
                }
            \right)
        \right]\\
        & = A\left(
            -\sin^2\theta
            - \sin\theta\sqrt{
                \sin^2\theta
                + B
            }
            + \cos^2\theta
            + \frac{
                \sin\theta\cos^2\theta
            }{
                \sqrt{
                    \sin^2\theta
                    + B
                }
            }
        \right)
    \end{aligned}
\end{equation}

Since $\max r(\theta) = r(\theta^*)$,
the derivative must be zero when $\theta=\theta^*$.

\begin{equation}
    \begin{aligned}
        \frac{
            dr
        }{
            d\theta
        }(\theta^*) = 0
        \implies & A\left(
            -\sin^2\theta^*
            - \sin\theta^*\sqrt{
                \sin^2\theta^*
                + B
            }
            + \cos^2\theta^*
            + \frac{
                \sin\theta^*\cos^2\theta^*
            }{
                \sqrt{
                    \sin^2\theta^*
                    + B
                }
            }
        \right) = 0\\
        \implies & -\sin^2\theta^*\cdot\sqrt{
            \sin^2\theta^*
            + B
        }
        - \sin\theta^*\cdot\left(
            \sin^2\theta^*
            + B
        \right)\\
        & + \cos^2\theta^*\cdot\sqrt{
            \sin^2\theta^*
            + B
        }
        + \sin\theta^*\cos^2\theta^* = 0\\
        \implies & \left(
            \cos^2\theta^*
            - \sin^2\theta^*
        \right)
        \sqrt{
            \sin^2\theta^*
            + B
        }
        = \sin^3\theta^*
        + B\sin\theta^*
        - \sin\theta^*\cos^2\theta^*\\
        \implies & \left(
            \cos^2\theta^*
            - \sin^2\theta^*
        \right)^2\left(
            \sin^2\theta^*
            + B
        \right)
        = \left(
            \sin^3\theta^*
            + B\sin\theta^*
            - \sin\theta^*\cos^2\theta^*
        \right)^2\\
        \implies & \sin^2\theta^*\cos^4\theta^*
        + B\cos^4\theta^*
        + \sin^6\theta^*
        + B\sin^4\theta^*
        - 2\sin^4\theta^*\cos^2\theta^*\\
        & - 2B\sin^2\theta^*\cos^2\theta^*
        = \sin^6\theta^*
        + B^2\sin^2\theta^*
        + \sin^2\theta^*\cos^4\theta^*
        + 2B\sin^4\theta^*\\
        & - 2B\sin^2\theta^*\cos^2\theta^*
        - 2\sin^4\theta^*\cos^2\theta^*\\
        \implies & B
        - 2B\sin^2\theta^*\cos^2\theta^*
        = B^2\sin^2\theta^*
        + 2B\sin^4\theta^*
    \end{aligned}
\end{equation}

Dividing both sides by $B$
and remembering that $\cos^2\theta^* = 1-\sin^2\theta^*$

\begin{equation}
    \begin{aligned}
        & 1
        - 2\sin^2\theta^*\left(
            1-\sin^2\theta^*
        \right)
        = B\sin^2\theta^*
        + 2\sin^4\theta^*\\
        \implies & 1
        - 2\sin^2\theta^*
        = B\sin^2\theta^*\\
        \implies & \theta^*
        = \arcsin\sqrt{
            \frac{
                1
            }{
                B
                + 2
            }
        }
    \end{aligned}
\end{equation}

Note that by setting $B=0$
we get $\theta^*=\frac{\pi}{4}$,
as expected from the case $y_0=0$.

\section{Conclusion}

The optimal take-off angle $\theta^*$
to maximize the projectile range
is given by

\begin{equation}
    \theta^* = \arcsin\sqrt{
        \frac{
            1
        }{
            \frac{
                2gy_0
            }{
                v_0^2
            }
            + 2
        }
    }
\end{equation}

For example, a baseball thrown with
initial velocity $v_0=20\frac{m}{s}$
from a $y_0=20m$ 5th floor window
will have an optimal take-off angle of
$\theta^* \approx 35.3^\circ$.

\end{document}
